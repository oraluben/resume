% !TEX program = xelatex

\documentclass{resume}

\usepackage{lastpage}
\usepackage{fancyhdr}
\usepackage{linespacing_fix} % disable extra space before next section
\usepackage[fallback]{xeCJK}

\setCJKfallbackfamilyfont{}{STSong}

\begin{document}
% \pagestyle{fancy}
% \fancyhf{}
\renewcommand\headrulewidth{0pt}
% \cfoot{\thepage\ of \pageref{LastPage}}

\name{严懿宸}

\basicInfo{
  \email{oraluben@outlook.com} \textperiodcentered\ 
  % \phone{(+86) 189-1807-5127} \textperiodcentered\
  \github[oraluben]{https://github.com/oraluben}
}

\section{工作经历}
\datedsubsection{\textbf{阿里云} ,上海}{2022.03 -- 现在}
\role{高级工程师}{针对架构的通用优化及CPython/Node.js 运行时优化}
% \begin{itemize}
%   \item{\url{https://github.com/alibaba/code-data-share-for-python}}
% \end{itemize}

\datedsubsection{\textbf{Oracle Labs} ,苏黎世,瑞士}{2020.09 -- 2021.02}
\role{实习}{Loom support for Graal Native Image (AOT compilation of coroutine in Java)}
\begin{itemize}
  \item{\url{https://github.com/oracle/graal/commits?author=oraluben}}
\end{itemize}

\datedsubsection{\textbf{PingCAP} ,远程}{2020.06 -- 2020.08, 2021.05 -- 2021.07}
\role{实习}{基于Fuzz技术提升TiDB数据库的可靠性。}
\begin{itemize}
	\item{设计并实现针对配置项的Fuzz工具Matrix。(2020)}
	\item{探索基于Fuzz技术测试TiDB SQL层的方法。(2021)}
\end{itemize}

\datedsubsection{\textbf{苏黎世联邦理工} ,苏黎世,瑞士}{2019.05 -- 2019.12}
\role{访问学生}{在Zhendong Su教授的指导下进行程序语言以及测试的研究。}
\begin{itemize}
	\item{2019秋季编译原理课程助教。}
\end{itemize}

%\datedsubsection{\textbf{Splunk},上海}{2017.07 -- 2018.01}
%\role{实习}{Performance Team}
%\begin{itemize}
%  \item{负责一部分功能开发以及测试用例编写;}
%  \item{引入基于覆盖率的单元/集成测试,发现了多个严重bug。}
%\end{itemize}

%\datedsubsection{\textbf{Intel},上海}{2015.06 -- 2016.09}
%\role{实习}{参与iLab的开发。iLab是Intel内部的一个基于Web的私有云系统。}
%\begin{itemize}
%  \item{参与Web/SDK的功能开发与修复;}
%  \item{设计了一个基于ActiveMQ的服务器通信协议,用于替代原有的RPC。}
%\end{itemize}

\section{教育经历}
\datedsubsection{\textbf{华东师范大学},上海}{2017.09 -- 2022.03}
  工学硕士学位,专业:软件工程

\datedsubsection{\textbf{华东师范大学},上海}{2013.09 -- 2017.06}
  工学学士学位,专业:软件工程

\section{项目}
% \datedsubsection{\textbf{Matrix}}{\url{https://github.com/chaos-mesh/matrix}}
% \role{Golang}{Matrix是一个根据特定DSL生成符合条件的配置文件进行模糊测试的工具。}
% \begin{itemize}
%   \item{根据需求设计DSL;}
%   \item{基于Golang实现工具。}
% \end{itemize}

\datedsubsection{\textbf{MC Fuzz}}{\url{https://tingsu.github.io/files/fse19-MCFuzz.pdf}}
\role{Python / C++ / Clang / Docker}{设计并实现了基于fuzzing的模型检查工具测试工具MC Fuzz。该工具以通用C程序作为输入,输出可以有效测试模型检查工具的代码。}
\begin{itemize}
\item{架构设计良好且易于扩展至不同的fuzzing策略/待测工具;}
\item{基于Docker的环境隔离。}
\end{itemize}

\datedsubsection{\textbf{Voile}}{\url{https://github.com/owo-lang/voile-rs}}
\role{Rust}{Voile是一个依赖类型(Dependent type)的编程语言,支持原生的Sum/Record以及Row polymorphism。}
\begin{itemize}
  \item{依赖类型的类型检查(Dependent type-checking);}
  \item{隐式参数的检查(Implicit argument checking)。}
\end{itemize}

% Reference Test
%\datedsubsection{\textbf{Paper Title\cite{zaharia2012resilient}}}{May. 2015}
%An xxx optimized for xxx\cite{verma2015large}
%\begin{itemize}
%  \item main contribution
%\end{itemize}

%% \section{\faHeartO\ 成就}
%% \datedline{}{Aug. 2017}

\section{技能}
\begin{itemize}[parsep=0.25ex]
  \item \textbf{编程语言}:
    熟悉Python,较为熟悉 C++/C\#/Java/Rust/OCaml/Go。

  % compiler theories
  \item \textbf{编译器}:
    熟悉编译器相关知识,有GraalVM和基于Clang/LLVM工具链开发的经历。
  
  \item \textbf{开发经历}:
    开发经验丰富,包括Web应用(Angular/Vue/PHP/Python),多种功能的客户端应用和在线游戏(WebSocket,HTML5)。
    
  \item \textbf{开发工具}:
    熟悉Git工作流,有包括YouTrack,Jira,Github等在内的项目管理工具的经验。熟悉Jetbrains,Visual Studio等IDE的使用。
  
  \item \textbf{交流与合作}:
    能够用英语进行流利的沟通,乐于与他人进行合作。
%
%  \item \textbf{软件开发}:
%    有能力发现并选择合适的方式解决开发过程中的问题。
\end{itemize}

\section{其他}
\begin{itemize}[parsep=0.25ex]
  \item 开源贡献:向GraalVM, LLVM/Clang, Python等项目提交过代码。
\end{itemize}

\section{论文}
\begin{itemize}[parsep=0.25ex]
  \item{Fully automated functional fuzzing of Android apps for detecting non-crashing logic bugs\\
  Su, T., \textbf{Yan, Y.}, Wang, J., Sun, J., Xiong, Y., Pu, G., Wang, K., Su, Z. \textsl{OOPSLA 2021}
  }
  \item{Finding and Understanding Bugs in Software Model Checkers\\
  Zhang, C, Su, T., \textbf{Yan, Y.}, Zhang, F., Pu, G., and Su, Z. \textsl{ESEC/FSE 2018}
  }
  \item{Smartunit: empirical evaluations for automated unit testing of embedded software in industry\\
  Zhang, C., \textbf{Yan, Y.}, Zhou, H., Yao, Y., Wu, K., Su, T., Miao, W. and Pu, G. \textsl{ICSE 2018 SEIP}
  }
\end{itemize}

%% Reference
%\newpage
%\bibliographystyle{IEEETran}
%\bibliography{mycite}
\end{document}
